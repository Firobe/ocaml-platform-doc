\documentclass{article}
\usepackage{float}
\usepackage{graphicx}
\usepackage{adjustbox}
\usepackage{booktabs}
\usepackage{hyperref}

\hypersetup{colorlinks=true} % Links are not framed but colored
\hypersetup{urlcolor=blue}

\title{Documentation overview of the OCaml platform}
\date{\today}

\newcommand{\tool}[1]{\texttt{#1}}
\newcommand{\good}{🟢}
\newcommand{\meh}{🟡}
\newcommand{\bad}{🔴}
\newcommand{\overall}{\textit{Overall official}}
%\newcommand{\bad}{\textcolor{green}{}}
\newcommand{\no}{~}
\newcommand{\official}[2]{\textbf{\href{#1}{#2}}}
\newcommand{\external}[2]{\href{#1}{#2}}
\newenvironment{doclist}
{\begin{minipage}[t]{0.2\textwidth}\begin{itemize}}
{\end{itemize}\end{minipage}}

\begin{document}
\maketitle
This is an overview of the available resources, both official and not, to
understand and use the main tools of the OCaml platform. The goal is to assess
the current state of the documentation, and decide what improvements should be
prioritized. The list tries to be exhaustive for official resources, but is not
exhaustive for non-official resources although it tries to include
significant/useful external resources (such as Real World OCaml).

The tools are roughly ordered in decreasing importance order (an understanding
of the tools near the top is absolutely necessary to be productive in OCaml).

All resources are roughly categorized according to the
\href{https://diataxis.fr/}{Diátaxis} documentation framework along the axes of
practial/theoretical and learning/working. Furthermore, for category a resource
can be rated \good,~\meh~or \bad~according to these (arbitrary, vague and
subjective) criteria:
\begin{itemize}
  \item \good~for resources that are complete, up-to-date and well-written;
  \item \meh~for resources that are not complete enough, or are austere to
    beginners;
  \item \bad~for resources that are out of date, or exist but with only the bare
    minimum of information, or are very hard to navigate.
\end{itemize}

Resources in \textbf{bold} are official ones: either written/published by the
authors of the tool or under the \url{https://ocaml.org} domain.

\begin{table}[htb]
    \centering
      \begin{tabular}{clccccl}
\toprule
Tool & Resource & Reference & Explanation & Tutorial & How-to guide & Notes \\
\midrule
% OPAM
 \tool{opam} &
 \official{https://opam.ocaml.org/doc/Manual.html}{Manual} &
 \good &
 \meh &
 \no &
 \no &
 Lacks a bit of structure\\
     &
 \official{https://opam.ocaml.org/doc/Usage.html}{Usage} &
 \no &
 \no &
 \bad &
 \no &
 Structured more like a reference than a good entry point\\
     &
 \official{https://opam.ocaml.org/doc/FAQ.html}{FAQ} &
 \no &
 \no &
 \no &
 \meh &
 \\
     &
 \official{https://ocaml.org/docs/up-and-running}{Up and running} &
 \no &
 \meh &
 \good &
 \no &
 Not hosted on the opam website: lacks visibility\\
 &\overall &
 \good &
 \meh &
 \meh &
 \meh &
 \textbf{Lack of clear entry point for beginners with concepts explained}\\
     &
 \external{https://dev.realworldocaml.org/platform.html}{Real World OCaml Ch.21} &
 \no &
 \good &
 \good &
 \no &
 \\

% DUNE
 \tool{dune} &
\official{https://dune.readthedocs.io/en/stable/index.html}{Manual} &
\good &
\meh &
\no &
\meh &
The structure is a bit all over the place\\
     &
\official{https://dune.readthedocs.io/en/latest/quick-start.html}{Quick start} &
\no &
\meh &
\good &
\no &\\
     &
\official{https://www.youtube.com/watch?v=BNZhmMAJarw}{Introduction video} &
\no &
\no &
\meh &
\no &
A bit outdated\\
 &\overall &
 \good &
 \meh &
 \meh &
 \meh &
 \textbf{The ocamlverse tutorial below should be official}\\
     &
\external{https://ocamlverse.net/content/quickstart_ocaml_project_dune.html}{OCamlverse
quickstart}&
\no &
\good &
\good &
\no &
\\
     &
 \external{https://dev.realworldocaml.org/platform.html}{Real World OCaml Ch.21} &
 \no &
 \meh &
 \meh &
 \no &
 \\
% MERLIN
 \tool{merlin} &
\official{https://github.com/ocaml/merlin/wiki}{Github Wiki} &
\bad &
\no &
\meh &
\no &List of commands not documented. User lost after editor setup.\\
   &
\official{https://ocaml.github.io/merlin/}{OCaml github.io} &
\meh &
\no &
\good &
\no &Nice style but lacks visibility!\\
 &\overall &
 \meh &
 \no &
 \good &
 \no &
 \textbf{The gihub.io should be linked from the wiki or somewhere visible. The
 wiki needs some structure.}\\
   &
\external{https://ocamlverse.net/content/editor_setup.html}{OCamlverse editor
setup} &
\no &
\no &
\meh &
\no &\\
% OCAML-LSP
 \tool{ocaml-lsp} &
\official{https://github.com/ocaml/ocaml-lsp}{README+docs/} &
\bad &
\bad &
\bad &
\no &No structure, information sometimes outdated, no editor setup instructions.\\
    &
\official{https://marketplace.visualstudio.com/items?itemName=ocamllabs.ocaml-platform}{OCaml
Platform Extension}&
\good &
\meh &
\good &
\no &Good for VS Code\\
 &\overall &
 \meh &
 \meh &
 \meh &
 \no &
 \textbf{Good if using VS Code (though the extension could be more visible).
 Other users are left on their own (or assuming merlin)}\\
% UTOP
 \tool{utop} &
 \official{https://github.com/ocaml-community/utop}{README} &
 \meh &
 \bad &
 \bad &
 \good &Good for installation, configuration. But you need to know how the
 general toplevel works.\\
      &
 \official{https://ocaml-community.github.io/utop/utop/index.html}{API} &
 \good &
 \no &
 \no &
 \no &\\
 &\overall &
 \good &
 \bad &
 \meh &
 \good &
 \textbf{Well documented but could link to resources explaining toplevel
 usage}\\
      &
\external{https://ocamlverse.net/content/toplevel.html}{OCamlverse
toplevel}&
 \no &
 \good &
 \meh &
 \no &Introduction to the generic toplevel\\

% OCAMLFORMAT
 \tool{ocamlformat} &
 \official{https://ocaml.org/p/ocamlformat/0.24.1/doc/index.html}{Main page} &
 \meh &
 \meh &
 \good &
 \good &\\
 &\overall &
 \meh &
 \meh &
 \good &
 \good &
 \textbf{The CLI manual is the only complete reference for options}\\
     &
 \external{https://dev.realworldocaml.org/platform.html}{Real World OCaml Ch.21} &
 \no &
 \no &
 \meh &
 \meh\\
% ODOC
 \tool{odoc} &
 \official{https://ocaml.github.io/odoc/}{Main page} &
 \good &
 \good &
 \good &
 \good &\textbf{Very well documented}\\
\bottomrule
        \end{tabular}
\end{table}

Furthermore, there are a few resources introducing the whole platform and the
interactions between some of the different tools:
\begin{itemize}
  \item \official{https://ocaml.org/docs/best-practices}{OCaml best practices}
  \item \official{https://ocaml.org/docs/up-and-running}{OCaml up and running}
  \item \external{https://dev.realworldocaml.org/platform.html}{Real World OCaml Ch.21}
  \item
    \external{https://lambdafoo.com/posts/2021-10-29-getting-started-with-ocaml.html}
    {Tim's "Getting Started with OCaml in 2021"}
\end{itemize}

Overall one of the biggest problems for beginners is the segmentation of the
documentation. Even amonst official resources and for the same tool,
documentation lives on:
\begin{itemize}
  \item \url{https://ocaml.org/docs/}
  \item \url{https://ocaml.github.io}
  \item \url{https://opam.ocaml.org}
  \item \url{https://dune.readthedocs.io}
  \item the Github pages of the tools
\end{itemize}

Except when reading a course (such as Real World OCaml), a beginner will most
likely learn about a single tool without seeing the big picture of the OCaml
tooling. The \href{https://ocaml.org/docs/platform}{OCaml platform} page gives
it blessing to a selection of tools but then leaves users on their own to figure
out how to install and use each tool.

\end{document}
